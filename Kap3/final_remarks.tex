\section{Final Remarks}
\label{sec:final_remarks_kap3}

In this chapter, several research works that aim to minimize CCO and improve MA were presented. The research works analysis revealed several facts. First, they introduce overhead that degrades the network performance or require substantial economic investments. Second, they do not optimize the probing interval by intelligent mechanisms intended to keep CCO and CUC within defined thresholds without compromising MA. Third, they do not consider the use of RL for SDN monitoring.\\

Unlike the works presented in this chapter, this master dissertation considers concepts from the KDN discipline for proposing an approach (\textit{cf.} Chapter~\ref{chapter:intelligent_probing}) that focuses on optimizing the probing interval regarding CCO and CUC.
%Finally, several research works that aim to network monitoring in SDN context were presented.\\

%As a conclusion, we could observe there is no definitive solution for network monitoring in SDN context yet. There are some efforts in the literature to cope with the different issues around with SDN network monitoring, but there is still a lot of work to do to achieve a mature solution proper for future SDN networks.