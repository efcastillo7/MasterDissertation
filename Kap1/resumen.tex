\textbf{\LARGE Resumen}\\\\
%\addcontentsline{toc}{chapter}{Abstract}\\\\
La monitorizaci\'on de tr\'afico ayuda a lograr la estabilidad de las redes al observar y cuantificar su comportamiento. Una soluci\'on de monitorizaci\'on de tr\'afico adecuada requiere la recopilaci\'on precisa y oportuna de estad\'isticas de flujo. Diversos investigadores han propuesto multiples enfoques para monitorear Redes Definidas por Software (Software-Defined Networks - SDN). Sin embargo, estos enfoques tienen algunas deficiencias. En primer lugar, no les preocupa el balance entre el intervalo de sondeo y la precisi\'on de monitoreo (Monitoring Accuracy - MA). En segundo lugar, carecen de mecanismos inteligentes destinados a optimizar este balance al aprender del comportamiento de la red. Esta disertaci\'on de maestr\'ia introduce un enfoque, llamado IPro, para abordar estas deficiencias. IPro est\'a formado por una arquitectura que sigue el paradigma de las Redes Definidas por el Conocimiento (Knowledge-Defined Networking - KDN), un algoritmo basado en el Aprendizaje por Refuerzo (Reinforcement Learning - RL) y un prototipo de IPro. En particular, IPro utiliza RL para determinar el intervalo de sondeo que mantiene dentro de umbrales (valores objetivo) la sobrecarga del canal de control (Control Channel Overhead - CCO) y el uso adicional de la CPU del controlador (CPU Usage of the Controller - CUC). Una extensa evaluaci\'on cuantitativa corrobora que IPro es un enfoque eficiente para el monitoreo de SDN con respecto a CCO, CCU y MA. \\ [2.0cm]

